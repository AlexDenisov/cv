\documentclass[11pt,a4paper]{moderncv}

\moderncvtheme[purple]{classic}
\usepackage[utf8]{inputenc}
\usepackage[scale=0.8]{geometry}

\usepackage[unicode]{hyperref}
\definecolor{linkcolour}{rgb}{0,0.2,0.6}
\hypersetup{colorlinks,breaklinks,urlcolor=linkcolour, linkcolor=linkcolour}

\firstname{Alex}
\familyname{Denisov}
\address{}{Kiev, Ukraine}
\mobile{+38 066 557 05 91}
\email{1101.debian@gmail.com}
\homepage{https://github.com/AlexDenisov}
\extrainfo{Software should be beautiful. Both inside and outside.}

\makeatletter
\renewcommand*{\bibliographyitemlabel}{\@biblabel{\arabic{enumiv}}}
\makeatother

\begin{document}
\maketitle

\section{Brief description}
\cvline
  {}
  {My main direction - iOS and RubyOnRails development.\newline{}
  This symbiosis allows me to create interacting system of a mobile client and a web service.\newline{}
  I really enjoy OpenSource software and do my best to develop and contribute OpenSource projects which I'm interested in.}

\section{OpenSource Projects}
\subsection{Objective-C}
\cvline
  {BloodMagic}
  {\url{https://github.com/railsware/BloodMagic}\newline{}
  BloodMagic is a framework, which gives you a way to create custom property attributes.}
\cvline
  {iActiveRecord}
  {\url{https://github.com/AlexDenisov/iActiveRecord}\newline{}
  SQLite-based ActiveRecord implementation for iOS.\newline{}
  Discontinued.}
\subsection{Swift}
\cvline
  {Sleipnir}
  {\url{https://github.com/railsware/Sleipnir}\newline{}
  BDD framework for Swift.}
\subsection{Ruby}
\cvline
  {guard-frank}
  {\url{https://github.com/AlexDenisov/guard-frank}\newline{}
  Ruby gem which allows to run Frank features for iOS automatically.
  \newline{}Supports Frank BDD framework.}
\cvline
  {guard-calabash-ios}
  {\url{https://github.com/AlexDenisov/guard-calabash-ios}\newline{}
  Ruby gem which allows to run Cucumber features for iOS automatically.
  \newline{}Supports Calabash-iOS BDD framework.}
\subsection{Ruby On Rails}
\cvline
  {GitLabHQ}
  {\url{https://github.com/gitlabhq/gitlabhq}\newline{}
  Open source software to collaborate on code.\newline{}
  Not an author, just sent few patches.}
\cvline
  {QuotePad}
  {\url{https://github.com/AlexDenisov/QuotePad}\newline{}
  Simple private service like \url{http://bash.org/}, which can be deployed to your own server.}


\section{Commercial Projects}
  \subsection{Objective-C}
  https://itunes.apple.com/us/app/talendo-jobborse-und-stellen/id763634231?mt=8
  \cvline
    {Talendo}
    {\url{https://itunes.apple.com/us/app/talendo-jobborse-und-stellen/id763634231?mt=8}\newline{}
    Development of iOS application.}
  \cvline
    {MyPickup Diary}
    {\url{https://itunes.apple.com/us/app/mypickup-diary/id503579078}\newline{}
    Development of iOS application with non-standart design.}
  \subsection{RubyOnRails}
  \cvline
    {Muddler}
    {\url{http://itunes.apple.com/us/app/muddler/id498512664}\newline{}
    Development of web backend for mobile clients.}
  \subsection{Objective-C/RubyOnRails}
  \cvline
    {My Lawyer}
    {\url{http://itunes.apple.com/ru/app/moj-urist/id475311463}\newline{}
    Development both iOS application and web backend.}

\section{Experience}
\cventry
  {May 2013 - Present}
  {Sotware engineer}
  {Railsware, Kiev, Ukraine}
  {}{}
  {Assessment of requirements for software development and time estimation. 
  \newline{}Software development.}
\cventry
  {Oct 2012 - Mar 2013}
  {iOS developer}
  {Stanfy LLC. Kiev, Ukraine}
  {}{}
  {Assessment of requirements for software development and time estimation. 
  \newline{}Software development.}
\cventry
  {Feb 2011 - Oct 2012}
  {iOS/RubyOnRails developer}
  {MLSDev LLC. Donetsk, Ukraine}
  {}{}
  {Assessment of requirements for software development and time estimation. 
  \newline{}Software development.
  \newline{}Web-services deployment and support.}
\cventry
  {Aug 2010 - Feb 2011}
  {C++/Qt developer}
  {WiseTroll (Sole Proprietorship). Donetsk, Ukraine}
  {}{}
  {Assessment of requirements for software development and time estimation. 
  \newline{}Software development.}


\section{Development skills}
\subsection{Programming Languages}
\cvline
  {Expert}{Objective-C}
\cvline
  {Intermediate}{C, C++, Ruby, Bash}
\cvline
  {Basic}{Swift, Perl, LaTeX}
\subsection{Frameworks and tools}
\cvline{iOS}{CocoaPods, Cedar, Calabash-iOS, RestKit, CoreData}
\cvline{RubyOnRails}{Cucumber, RSpec, Rabl, Devise, CanCan, Twitter Bootstrap, Grape, Slim}

\section{Server administration skills}
\cvline
  {Redmine, GitLabHQ}{Deployment, support and migration to another servers (Debian, Ubuntu)}
\cvline
  {Jenkins}{Deployment and support on MacOS X}
\cvline
  {Nginx, Apache}{Setup and configuration on rpm- and deb-based linux systems}

\section{Computer skills}
  \cvline
  {IDE/Editors}{Xcode, AppCode, QtCreator, Vim}
  \cvline
  {OS}{Linux (RedHat, Debian), MacOS X}
  \cvline
  {VCS}{Git, Mercurial}
  \cvline
  {Development technics}{BDD, TDD, XP, Continuous Integration}

\section{Education}
  \cventry
    {Oct 2006 - Dec 2011}
    {Software Development}
    {Computer Academy "Step", Donetsk, Ukraine}
    {}{}
    {Team member, represented Ukraine at the final of ImagineCup'10 at Poland.}
  \cventry
    {Sep 2006 - Jul 2010}
    {Management of Organizations, Bachelor}
    {Open International University of Human Development “Ukraine”}
    {}{}{}

\section{Public activity}
  \cvline{LinkedIn}{\url{http://www.linkedin.com/pub/alexey-denisov/2a/bab/a29}}
  \cvline{GitHub}{\url{https://github.com/AlexDenisov}}
  \cvline{CoderWall}{\url{http://coderwall.com/AlexDenisov}}
  \cvline{Habrahabr}{\url{http://habrahabr.ru/users/1101_debian/}}
  \cvline{RubyGems}{\url{https://rubygems.org/profiles/58746}}
  \cvline{BlogPosts}{\url{http://railsware.com/blog/author/alexey-denisov/}}
  \cvline{Speakerdeck}{\url{https://speakerdeck.com/alexdenisov}}

\end{document}
